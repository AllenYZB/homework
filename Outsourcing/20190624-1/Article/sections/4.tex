\section{程序运行结果示例}
\subsection{数据特征分析与预测}
我们可以通过上述思想编制程序,从而得出以下的结论(得分均按照\verbbox[violet]{r2_score}):
\begin{enumerate}
    \item \verbbox[gray]{day.csv}数据集中:
        \begin{itemize}
            \item 未注册用户预测最佳模型:\verbbox[violet]{GradientBoostingRegressor},分数为0.83;
            \item 注册用户预测最佳模型:\verbbox[violet]{GradientBoostingRegressor},分数为0.75;
            \item 总用户预测最佳模型:\verbbox[violet]{GradientBoostingRegressor},分数为0.78。
        \end{itemize}

    \item \verbbox[gray]{hour.csv}数据集中:
        \begin{itemize}
            \item 未注册用户预测最佳模型:\verbbox[violet]{BaggingRegressor},分数为0.81;
            \item 注册用户预测最佳模型:\verbbox[violet]{ExtraTreesRegressor},分数为0.88;
            \item 总用户预测最佳模型:\verbbox[violet]{ExtraTreesRegressor},分数为0.89。
        \end{itemize}

    \item 时间相关变量数据集中:
        \begin{itemize}
            \item 未注册用户预测最佳模型:\verbbox[violet]{GradientBoostingRegressor},分数为0.68;
            \item 注册用户预测最佳模型:\verbbox[violet]{GaussianProcessRegressor},分数为0.82;
            \item 总用户预测最佳模型:\verbbox[violet]{GaussianProcessRegressor},分数为0.79。
        \end{itemize}
\end{enumerate}


\subsection{数据可视化}
详见问题分析章节中数据可视化示例小节。
