\section{数据预处理}
数据往往因缺乏统一标准与结构导致杂乱性,因对客观事物在数据库中存在两个以上的描述导致重复性,因信息模糊、随机与缺失导致不完整性。因此为增加模型性能,简化模型计算,我们需要对数据进行预处理。\cite{撇味大白菜2018机器学习笔记:特征工程与数据降维,刘明吉2000数据挖掘中的数据预处理}
\subsection{数据清洗}
数据清洗主要包括如下部分:
\begin{enumerate}
    \item\label{P:缺失值} 缺失值处理;
    \item\label{P:噪声} 噪声数据处理;
    \item\label{P:异常值} 异常值处理;
    \item\label{P:脏数据} 脏数据处理;
    \item\label{P:去重} 去重处理。
\end{enumerate}

根据数据集的探索我们可以得出,原数据不存在缺失值,并且结构完善与统一,无重复数据,所以我们只需要进行第~\ref{P:噪声}部分与第~\ref{P:异常值}部分的数据清洗即可。

通过主观判断与可视化判断,数据虽然存在被随机误差予离群值,但是数据取值范围正常却数量远小于数据总量,所以本文并未对数据进行严格的噪声数据处理与异常值数据处理,这将补充到第~\ref{S:future-work}节的未来工作当中。


\subsection{数据变换}
数据变换一般包括如下部分\cite{撇味大白菜2018机器学习笔记:特征工程与数据降维}:
\begin{enumerate}
    \item 平滑:去掉数据中的噪音。
    \item 聚集:为多粒度数据分析构造数据。
    \item 数据泛化:用高层次概念代替低层次原始数据。
    \item 规范化:将属性数据按比例缩放,使之落入一个特定区间。
    \item 属性构造:构造新的属性并添加到属性集中。
\end{enumerate}

根据数据集的探索我们可以得出,原数据已经进行了数据泛化与规范化处理,所以本文并未对数据进行数据泛化、规范化等数据变换过程,这将补充到第~\ref{S:future-work}节的未来工作当中。


\subsection{数据简化}
数据简化,又称数据规约,指在尽可能保持数据原貌的前提下,最大限度地精简数据量。同时,用于数据简化的时间不应当超过之后在数据挖掘中节省的时间\cite{撇味大白菜2018机器学习笔记:特征工程与数据降维}。数据简化有以下三种常见方法 :
\begin{enumerate}
    \item 维度规约;
    \item 数量规约;
    \item 数据压缩。
\end{enumerate}

本文主要是用维度规约,从高维特征空间向地位特征空间进行映射,即降维。一般来说,现实生活中所得到的数据具备共线性、稀疏性等特征,因此对数据进行降维有助于减少特征属性之间的个数,同时确保特征属性之间是相互独立的。降维主要有以下方法(均来自\verbbox[violet]{sklearn.decomposition}模块)\cite{scikit-learn,sklearn_api,DataAnalyst2019机器学习之sklearn中的降维算法,Cyrille2015sne}:

\begin{itemize}
    \item 主成分分析:
        \begin{itemize}
            \item \verbbox[violet]{PCA};
            \item \verbbox[violet]{IncrementalPCA};
            \item \verbbox[violet]{KernelPCA};
            \item \verbbox[violet]{MiniBatchSparsePCA};
            \item \verbbox[violet]{SparsePCA};
            \item \verbbox[violet]{TruncatedSVD}.
        \end{itemize}
    \item 因子分析:
        \begin{itemize}
            \item \verbbox[violet]{FactorAnalysis}.
        \end{itemize}
    \item 独立成分分析:
        \begin{itemize}
            \item \verbbox[violet]{FastICA}.
        \end{itemize}
    \item 字典学习:
        \begin{itemize}
            \item \verbbox[violet]{DictionaryLearning};
            \item \verbbox[violet]{MiniBatchDictionaryLearning}.
        \end{itemize}
    \item 高级矩阵分解:
        \begin{itemize}
            \item \verbbox[violet]{LatentDirichletAllocation};
            \item \verbbox[violet]{NMF};
            \item \verbbox[violet]{SparseCoder}.
        \end{itemize}
\end{itemize}

本文主要采用\verbbox[violet]{PCA}与\verbbox[violet]{TruncatedSVD},通过实际的计算发现,选取阈值为95\%以上的\verbbox[violet]{TruncatedSVD}降维方法所得的效果最好,即$R^2$相对较高。所以在接下来的构造回归模型中,我们采用该降维方法对数据进行处理。
