\section{回归模型的构造}
\subsection{评价与误差函数}
我们选择以下评价函数与误差函数来评定模型的优劣(均来自于\verbbox[violet]{sklearn.metrics}模块):
\begin{itemize}
    \item 评价函数:
        \begin{itemize}
            \item \verbbox[violet]{explained_variance_score};
            \item \verbbox[violet]{r2_score}.
        \end{itemize}
    \item 误差函数:
        \begin{itemize}
            \item \verbbox[violet]{max_error};
            \item \verbbox[violet]{mean_absolute_error};
            \item \verbbox[violet]{mean_squared_error};
            \item \verbbox[violet]{mean_squared_log_error};
            \item \verbbox[violet]{median_absolute_error}.
        \end{itemize}
\end{itemize}

广泛地对模型进行评价,同时加上模型的训练时间与预测时间,多维度地评价模型的优劣。


\subsection{回归模型的选择}
广泛地训练模型,从而取效果最好的,因此我们采用以下的回归模型\cite{scikit-learn,sklearn_api}:
\begin{itemize}
    \item \verbbox[violet]{sklearn.ensemble}:
        \begin{itemize}
            \item \verbbox[violet]{AdaBoostRegressor};
            \item \verbbox[violet]{BaggingRegressor};
            \item \verbbox[violet]{ExtraTreesRegressor};
            \item \verbbox[violet]{GradientBoostingRegressor};
            \item \verbbox[violet]{IsolationForest}.
        \end{itemize}
    \item \verbbox[violet]{sklearn.gaussian_process}:
        \begin{itemize}
            \item \verbbox[violet]{GaussianProcessRegressor}.
        \end{itemize}
    \item \verbbox[violet]{sklearn.linear_model}:
        \begin{itemize}
            \item \verbbox[violet]{ElasticNetCV};
            \item \verbbox[violet]{HuberRegressor};
            \item \verbbox[violet]{LassoLarsCV};
            \item \verbbox[violet]{LogisticRegression};
            \item \verbbox[violet]{PassiveAggressiveRegressor};
            \item \verbbox[violet]{RANSACRegressor};
            \item \verbbox[violet]{RidgeCV};
            \item \verbbox[violet]{SGDRegressor};
            \item \verbbox[violet]{TheilSenRegressor}.
        \end{itemize}
    \item \verbbox[violet]{sklearn.neighbors}:
        \begin{itemize}
            \item \verbbox[violet]{KNeighborsRegressor};
            \item \verbbox[violet]{RadiusNeighborsRegressor}.
        \end{itemize}
    \item \verbbox[violet]{sklearn.neural_network}:
        \begin{itemize}
            \item \verbbox[violet]{MLPRegressor}.
        \end{itemize}
\end{itemize}

最后生成一个数据结构如下的结果:

\tikz [font=\texttt\footnotesize,
    grow=right, level 1/.style={sibling distance=3em,level distance=.1\textwidth},
                level 2/.style={sibling distance=3em,level distance=.15\textwidth},
                level 3/.style={sibling distance=1.5em,level distance=.2\textwidth},
                level 4/.style={sibling distance=1em,level distance=.25\textwidth},]
    \node{Result}
        child { node{cnt}
            child { node{\dots} }
        }
        child { node{casual}
            child { node{\dots} }
        }
        child { node{registered}
            child { node{info}
                child { node{\vdots} }
                child { node{AdaBoostRegressor}
                    child { node{r2\_score} }
                    child { node{fit\_time} }
                    child { node{pred\_time} }
                }
            }
            child { node{best}
                child { node{model} }
                child { node{r2\_score} }
            }
        };

因此我们可以通过上述思想编制程序,从而得出以下的结论(得分均按照\verbbox[violet]{r2_score}):
\begin{enumerate}
    \item \commandbox{day.csv}数据集中:
        \begin{itemize}
            \item 未注册用户预测最佳模型:\verbbox[violet]{GradientBoostingRegressor},分数为0.83;
            \item 注册用户预测最佳模型:\verbbox[violet]{GradientBoostingRegressor},分数为0.75;
            \item 总用户预测最佳模型:\verbbox[violet]{GradientBoostingRegressor},分数为0.78。
        \end{itemize}

    \item \commandbox{hour.csv}数据集中:
        \begin{itemize}
            \item 未注册用户预测最佳模型:\verbbox[violet]{BaggingRegressor},分数为0.81;
            \item 注册用户预测最佳模型:\verbbox[violet]{ExtraTreesRegressor},分数为0.88;
            \item 总用户预测最佳模型:\verbbox[violet]{ExtraTreesRegressor},分数为0.89。
        \end{itemize}

    \item 时间相关变量数据集中:
        \begin{itemize}
            \item 未注册用户预测最佳模型:\verbbox[violet]{GradientBoostingRegressor},分数为0.68;
            \item 注册用户预测最佳模型:\verbbox[violet]{GaussianProcessRegressor},分数为0.82;
            \item 总用户预测最佳模型:\verbbox[violet]{GaussianProcessRegressor},分数为0.79。
        \end{itemize}
\end{enumerate}
