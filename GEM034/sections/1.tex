\section{《美的历程》相关摘抄}
\subsection{明清文艺思潮}
\subsubsection{市民文艺}
纵观前面,如可说汉代文艺反映了事功、行动,魏晋风度、北朝雕塑表现了精神、思辨,唐诗宋词、宋元山水展示了襟怀、思绪,那末,以小说戏曲为代表的明清文艺所描绘的却是世俗人情。这是又一个广阔的对象世界,但已不是汉代艺术中的自然征服,不是那古代蛮勇力量的凯旋,而完全是近代市井的生活散文,是一幅幅平淡无奇却五花八门、多姿多彩的社会风习图画。

文艺毕竟走在前头,开时代风气之先。在宋代平话,就已有所谓“烟粉”、“灵怪”、“传奇”、“公案”以及“讲史”等等类别,说明这种以广大市民为对象的近代说唱文学已拥有广阔的题材园地。它与六朝志怪或唐人小说已经很少相同了。它不是以单纯的猎奇或文笔的华丽来供少数贵族们思辨或阅读,而是以描述生活的真实来供广大听众消闲取悦。尽管从文词的文学水平和成就看,似乎并无可取,然而,其实际的艺术效采却相当可观,应该说已经超过了以前任何贵族文艺。

这种世俗文学的审美效果显然与传统的诗词歌赋,有了性质上重大差异,艺术形式的美感逊色于生活内容的欣赏,高雅的趣味让路于世俗的真实。这条文艺河谷发展到明中叶,便由涓涓细流汇为江湖河海。由口头的说唱发展为正式的书面语言。以《喻世明言》、《警世通言》、《醒世恒言》和初二刻《拍案惊奇》为代表,标志着这种市民文学所达到的繁荣顶点,具有了自己的面貌、性格和特征,对近代影响甚巨。它们的选本《今古奇观》便流传三百余年而历久不衰。正如这个选本的序言所说,这些作品确乎是“极摹人情世态之歧,备写悲欢离合之致”,把当时由商业繁荣所带给封建秩序的侵蚀中的社会作了多方面的广泛描绘。多种多样的人物、故事、情节都被揭示展览出来,尽筲它们像汉代浮雕似地那样薄而浅,然而它所呈现给人们的,却已不是粗线条勾勒的神人同一、叫人膜拜的古典世界,而虽有现实人情味的世俗日常生活了。对人情世俗的津津玩味,对荣华富贵的钦羡渴望。对性的解放的企望欲求,对“公案”、神怪的广泛兴趣,……尽管这里充满了小市民种种庸俗、低级,浅薄无聊,尽管这远不及上层文人士大夫艺术趣味那么高级、纯粹和优雅,但它们倒是有生命活力的新生意识,是对长期封建王国和儒学正统的侵袭破坏。它们有如《十日谈》之类的作作品出现于欧洲文艺复兴时代一样。

其中一个流行而突出的题材或主题。是普通男女之间的性爱。这种题材在唐诗和以前文艺中并无重要地位,在宋词中则主要是作为与勾栏妓女有关的咏叹(例如柳永的某些作品),但已幵始表现出某种平等而真挚的男女情爱,特别是青年女性对爱情的热情、留恋、执宥和忠诚,得到了肯定性的抒写描画,反映出妇女不只是作为贵族们的玩物,而有了人的地位。随宥商业经济空前发达和城市生活的高度繁荣,自然生理的性爱题材日益取得社会性的意义和内容,自應的、平等的、互爱的男女情热,具有冲破重重封建礼俗去争取自由的价值和意义。或者是一见倾心而至死不渝,或者是历经曲折终成眷属,或者是始乱终弃结局悲惨,或者是肉欲横流追求淫荡,从《卖油郎独占花魁》、《杜十娘怒沉百宝箱》到《乔太守乱点弩鸯谱》、《玉堂春落难寻夫》到《任君用恣乐深闺》……,形形色色,五光十彩。其中,有对献身纯真爱情的歌颂赞扬,有对封建婚姻的讽刺嘲笑,有对负心男子的鞭挞谴责,也有对色情荒淫的欣赏玩味……。总之,这里的思想、意念、人物、形象、题材、主题,已大不同于封建文艺和文人士大夫的传统。它既来源于说唱文学,满足的对象是一般“市井小民”也就使它成为世俗生活的风习画廊。在这个画廊中,男女性爱并非唯一主题,所展开的是世俗生活的多方面,这里有公正的义士、善良的武生,有贪婪残暴的县丞、奸邪阴险的权贵,……由于社会开始孕育着从封建母胎里的解怀、个人的际遇、遭逢,前途和命运逐渐失去独一无二的原有模式,各色人物都在为自己奋斗,或经商致富,或投考中举,或白首穷经竟一无所获,或巧遇良机而顿致富贵。一方曲是追求,另一方面是机遇,封建秩序的削弱、阶级关系的变迁使现实社会中个人道路的多样化趋势在萌芽,使现实生活的偶然性与必然性的关系更为丰富而复杂。虽然还谈不上个性解放,但在这些世俗小说中已可窥见对个人命运的关注。从思想意识说,这里有对邪恶的唾骂和对美徳的赞扬,同时也有对宿命的宣扬和对因果报应,逆来顺受的渲染。总之某种近代现实性世俗性与腐朽庸俗的传统落后意识渗透、交错与混合,是这种初兴市民文学的一个基本特征。这里没有远大的思想。深刻的内容,也没有真正抱负雄伟的主角和突出的个性,激昂的热情。它们是一些平淡无奇然而却比较真实和丰富的世俗的或幻想的故事。

由于它们由说唱演化而来。为了满足听众的要求,重视情节的曲折和细节的丰富,成为这一文学在艺术上的重要发展。具有曲折的情节吸引力量和具有如临其境如见其人的细节真实性。构成说唱者及其作品成败的关键。从而如何构思、选择、安排情节,使之具有戏剧性,在人意中又出人意外;如何概括地模拟描写事物,听来逼真而又不嫌烦琐;不是去追求人物性格的典型性而赶追求情节的合理、述说的逼真,不是去刻画事物而是去重视故事,在人情世态、悲欢离合的场合境遇中,显出故事的合理和真实来引人人胜,便成为目标所在。也正是这些奠定了中闻小说的民族风格和艺术特点。

与宋明话本、拟话本并行发展的是戏曲。元代少数民族入主中原造成了经济、文化的倒退,却也创造了文人士大夫阶层与民间文学结合的环境。它的成果就是反映生活。内容丰满的著名的元代杂剧。关汉卿、王实甫、白朴、马致远四大家成为一代文学正宗,《窦娥冤》、《西厢记》,《墙头马上》等等成为至今流传的传统剧目。到明中叶以后,传奇的大量涌现,把戏曲推上一个新的阶段。除了文学上的意义外,更重要的是,它已发展和定形为一种由说唱、表演、音乐、舞蹈相结合的综合艺术,创造了中国民族持色的戏曲形式的艺术美。直到昆曲和京剧,在所谓唱、念、做、打中。把这种美推到了炉火纯青无与伦比的典范高度。像昆曲,以风流潇洒,多情善感的小生。小旦为主角,以精工细作的姿态唱腔来刻划心理、情意,配以优美文词,相当突出地表现了一代风神。

这是一种经过高度提炼的美的楮华。千锤百炼的唱腔设计、一举手一投足的舞蹈化的程式动作,離塑性的亮相、象征性、示意性的环境布置,异常简洁明了的情节交代,高度选择的戏剧冲突(经常是能激起巨大心理反响的伦理冲突),使内容和形式交融无间,而特别突出了积淀了内容要求的形式美。这已不是简单的均衡对称,变化统一的外在形式美,而是与内容、意义交织在一起。如京剧的吐字,就不光是一个外在形式美问题。而且要求与内容含义的表达有所交融(所谓“声情”与“词情”等等)。但其中,外在形式美又仍然占有极重要的地位。中国戏曲尽管以再现的文学剧本为内容,却通过音乐、舞蹈、唱腔、表演,把作为中国文艺的灵魂的抒情特性和线的艺术,发展到又一个空前绝后、独一无二的综合境界。它实际上并不以文学内容而是以艺术形式取胜,也就是说以美取胜。

能不对昆曲,京剧中那种种优芙的唱段唱腔心醉动怀?能不对那袅袅轻烟般的出场入场、连行程也化为S形的优雅动作姿态叹为观止?高度提炼、概括而又丰富具体,已经程式化而又仍有—定个性,它不是—般形式美,而正是“有意味的形式”。尽管进入上层和宫廷之后,趣味日见纤细,但它的基础仍是广泛的“市井小民”,它仍属于市民文艺的一部分。