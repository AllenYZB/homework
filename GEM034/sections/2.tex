\section{如何探索多维度的科学与艺术}
人们观察、思考与表述某事物的“思维角度”,简称“维度”。而每一个维度都积累着自事物诞生时到现在的所有知识,并且百年外的积累已经历了时间的筛选,将最最精华的部分沉淀在历史长河中;而百年内则是百家争鸣,不断有新的知识对原有框架进行补充完善,同时不断也会有新的突破与瓶颈。因此想穷尽所有维度去探索科学与艺术是不现实的,人的精力毕竟有限,面面俱到则面面不到,因此首先需要有自己的方法论,并加以运用与发挥。

就拿中国戏曲为例,我们可以将戏曲的学习看作一个图,从戏曲元素划分我们有九大维度可以涉及(台大开放式课程):故事、诗歌、音乐、舞蹈、杂技、演员角色人物三位一体、狭隘剧场、代言体与说唱文学的叙述方法,而这九大维度下又有各自若干小维度,以此类推。所以我们有两种纯方式可以学习中国戏曲:广度优先,即所有维度统一步伐、相辅相成;深度优先,即各个维度逐一击破、脚踏实地。广度优先与深度优先各有优缺点,广度优先虽然可以快速地了解整个中国戏曲,从宏观上有一定的认识,但是所有维度统一学习需要消耗大量心血且不利于深入;深度优先虽然可以快速地了解中国戏曲的某些方面,但是缺少整体的把握不利于取舍维度。因此我们可以将两种纯方法进行组合,趋利避害。

以我之所学及所见,探索多维度的中国戏曲,可以先采用广度优先,从宏观上把握整体,同时将所学内容相互比较相辅相成,一生二,二生三。接下来可以采用深度优先,对感兴趣的方面进行深入的学习与研究,脚踏实地地增加对戏曲维度的理解。

以上方法论是有前提假设的,即假定戏曲的学习这张图是静态的。但是在实际情况中,任何事物都是发展的,我们应该以发展的眼光看待问题,所以如何确定古今知识的优先级就变得尤为重要了。一方面古代的知识是经过时间沉淀积累的,一大段时间才凝聚出一小段历史,一大段历史才能凝聚出一小段文化,文化之重自古使然;另一方面不可否认的是,历史上的知识可能只是当时认知条件下的真理,而中国戏曲在当今世界中仍在健壮地发展着,甚至会有颠覆历史认知的改变与时俱进。因此古今知识孰重孰轻便是在探究多维度中国戏曲上面是不可回避的问题。

“遵循古典、但不因循古典。利用现代,但不滥用现代。”

在台大的开放式课程中也提到了一个有趣的结论:我们现在对戏曲的认识已经超过古人很多了,而且更具备层次性。因此我们可以从自己出发,反视古人,想为什么他们只能想到这里,达到这样的境界。换一个角度看问题往往会得到出人意料的结果,受此启发,我们便可借今人的层次结构,将古人的创作划分进现有的维度中来,分而治之。毕竟只有一套脱于实践理论,理论只能是理论;相反如果得到很好的结果,那便是在现有知识下的真理。

因此探索包含时间维度的中国戏曲,我们可以先采用现代的维度划分法,并实践在古人的作品之中,在现有的维度上面将理论知识转化为实践真知,不断积累记录。待到现代的维度划分法不起作用时,就回观现代的知识,通过别人的已有理论加上自己的实践相结合,构造属于自己的理论架构。此所谓古今兼用,互补所长,相互发展。

最后,要想真正探索出多维戏曲,尚需反观历史,从历史的角度寻到戏曲产生的合理性,从而推断出其内在情感倾向。值得一提的是元代少数民族入主中原不仅造成了经济文化的倒退,也为文人士大夫阶层与民间文学结合提供了社会环境,想要了解内容丰满的元代杂剧及四大家关汉卿、王实甫、白朴、马致远及其至今流传的传统剧目,便需反观历史,找出其重要的创举及其意义,如此方能领会中国戏曲中所蕴含的美学意义。

中国戏曲高度提炼、概括而又丰富具体,已经程式化而仍有一定的个性,中国戏曲不是一般形式美,而是“有意味的形式”,所以在探索多维度的中国戏曲中,不能只拘泥于理论上的框架方法,更需要用心去感受,敢于发散思维。
