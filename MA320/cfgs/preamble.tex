% !Mode:: "TeX:UTF-8"
% !TEX program  = xelatex

% Hyperref
    % hyperref
    \usepackage{hyperref}
        \hypersetup{
            pdfauthor={Iydon},
            pdfcreator={Microsoft® Word 2017},
            pdfproducer={Microsoft® Word 2017},
        }

% Math
    % symbols and environment
    \usepackage{amsmath,amssymb}
        \newcommand{\dd}{\mathrm{d}}
        \newcommand{\diff}[2]{\frac{\dd #1}{\dd #2}}
        \newcommand{\tdiff}[2]{\tfrac{\dd #1}{\dd #2}}
        \newcommand{\RR}{\mathbb{R}}
        \newcommand{\vecw}{\symbf{w}}
        \newcommand{\vecx}{\symbf{X}}
        \newcommand{\vecy}{\symbf{Y}}
        \newcommand{\logit}{\mathrm{logit}}
    % font
    \usepackage{unicode-math}
        \unimathsetup{math-style=ISO, bold-style=ISO, mathrm=sym}
        \setsansfont{FiraGO}[BoldFont=* SemiBold, Numbers=Monospaced]
        \setmathfont{Fira Math Regular}

% Header and Footer
    % geometry
    \usepackage{geometry}
        % \geometry{margin=1in}
    % fancy header
    \usepackage{fancyhdr}
        % \fancyhead[L]{}
        % \fancyhead[C]{Iydon Liang, 11711217}
        % \fancyhead[R]{}
        % \renewcommand{\headrulewidth}{0.4pt}
        % \pagestyle{fancy}

% Listings
    % Python
    \usepackage{tcolorbox}
        \tcbuselibrary{listings,breakable}
        \newtcblisting{pylist}[1]{
            listing options={language=Python,numbers=left,
                numberstyle=\tiny\color{red!75!black},breaklines,
                basicstyle=\small,keywordstyle=\color{blue!85!black},
                commentstyle=\color{white!75!black}\textit},
            title=\texttt{#1},listing only,breakable,
            left=6mm,right=6mm,top=2mm,bottom=2mm,
            colback=orange!5!white,colframe=orange!75!black}
        \newtcbinputlisting{\Python}[2]{%
            listing options={language=Python,numbers=left,
                numberstyle=\tiny\color{red!75!black},
                basicstyle=\small,keywordstyle=\color{blue!85!black},
                commentstyle=\color{white!75!black}\textit,breaklines},
            listing only,breakable,
            listing file={#2},title=\texttt{#1},
            left=6mm,right=6mm,top=2mm,bottom=2mm,
            colback=orange!5!white,colframe=orange!75!black}
        \tcbuselibrary{xparse}
            \DeclareTotalTCBox{\verbbox}{ O{orange} v !O{} }
                {fontupper=\ttfamily,nobeforeafter,tcbox raise base,
                 arc=0pt,outer arc=0pt,top=0pt,bottom=0pt,left=0mm,
                 right=0mm,leftrule=0pt,rightrule=0pt,toprule=0.3mm,
                 bottomrule=0.3mm,boxsep=0.5mm,bottomrule=0.3mm,boxsep=0.5mm,
                 colback=#1!10!white,colframe=#1!50!black,#3}{#2}
        \DeclareTotalTCBox{\commandbox}{ s v }
                {verbatim,colupper=white,colback=black!75!white,colframe=black}
                {\IfBooleanTF{#1}{\textcolor{red}{\ttfamily\bfseries >> }}{}
                    \lstinline[language=sh,morekeywords={python,python2,python3},keywordstyle=\color{blue!35!white}\bfseries]^#2^}
        \tcbuselibrary{theorems}
        \newtcbtheorem{theobox}{Theorem}%
            {colback=white,colframe=orange!90!black,fonttitle=\bfseries,%
             arc=2mm,separator sign={\ $\blacktriangleright$}}{theo}%
        \newtcbtheorem{defnbox}{Definition}%
            {colback=white,colframe=blue!90!black,fonttitle=\bfseries,%
             arc=2mm,separator sign={\ $\blacktriangleright$}}{defn}%
    % Notepad
    \tcbuselibrary{most}
    \usetikzlibrary{shapes.geometric}
    \definecolor{blue1}{RGB}{176,209,244}
    \definecolor{gray1}{RGB}{212,219,239}
    \definecolor{gray2}{RGB}{239,239,239}
    \definecolor{red1}{RGB}{187,71,54}
    \newtcolorbox{notepad}[1][Untitled]{
    freelance,
    coltitle=black,
    fonttitle=\footnotesize,
    frame code={
      \draw[rounded corners=2pt,fill=blue1] 
        ([xshift=-2pt]frame.north west) --
        ([xshift=2pt]frame.north east) --
        ([xshift=2pt,yshift=-2pt]frame.south east) --
        ([xshift=-2pt,yshift=-2pt]frame.south west) -- cycle;
      \node[draw,fill=red1,anchor=north east,inner ysep=0pt,text width=10pt,minimum height=8pt] 
      at ([xshift=-1.5pt]frame.north east) (close) {};
      \node[draw,fill=gray1,anchor=north east,inner ysep=0pt,text width=10pt,minimum height=8pt] 
      at ([xshift=\pgflinewidth]close.north west) (minim) {};
      \node[draw,fill=gray1,anchor=north east,inner ysep=0pt,text width=10pt,minimum height=8pt] 
      at ([xshift=\pgflinewidth]minim.north west) (hide) {};
      \draw[double,fill=gray1] 
        ([xshift=-2pt,yshift=-2pt]minim.center) rectangle 
        ([xshift=2pt,yshift=2pt]minim.center);
      \draw[fill=white] 
        ([xshift=-3pt,yshift=-1pt]hide.center) rectangle 
        ([xshift=3pt,yshift=-2.2pt]hide.center);
      \draw[fill=white,rotate=45] 
        ([xshift=-3pt,yshift=-0.8pt]close.center) rectangle 
        ([xshift=3pt,yshift=0.8pt]close.center);
      \draw[fill=white,rotate=135] 
        ([xshift=-3pt,yshift=-0.8pt]close.center) rectangle 
        ([xshift=3pt,yshift=0.8pt]close.center);
      \draw[draw=none,fill=white,rotate=45] 
        ([xshift=-3pt+\pgflinewidth,yshift=-0.8pt+\pgflinewidth]close.center) rectangle 
        ([xshift=3pt-\pgflinewidth,yshift=0.8pt-\pgflinewidth]close.center);
      },
      title code={
        \draw[fill=gray1] 
          (title.south west) rectangle 
          ([yshift=10pt]title.south east);
        \node[anchor=south west,inner ysep=0pt,xshift=1.5pt] 
          at (title.south west) 
          {\footnotesize File\quad Edit\quad Format\quad View\quad Help};
      },
      bottomtitle=12pt,
    interior titled code={
      \draw[fill=white] 
        (interior.north west) --
        (interior.north east) --
        (interior.south east) --
        (interior.south west) -- cycle;
      \draw[fill=gray2,draw=gray!30] 
        ([xshift=-7pt,yshift=-\pgflinewidth]interior.north east) rectangle 
        ([xshift=-\pgflinewidth,yshift=7+\pgflinewidth]interior.south east);    
      \draw[fill=gray2,draw=gray!30] 
        ([yshift=7pt,xshift=\pgflinewidth]interior.south west) rectangle 
        ([xshift=-7pt-\pgflinewidth,yshift=+\pgflinewidth]interior.south east);
      \node[isosceles triangle,fill=black!70,minimum height=1cm,minimum width=2cm, shape border rotate=180, isosceles triangle stretches,scale=0.105] 
        at ([yshift=3.5pt,xshift=4pt]interior.south west) {};      
      \node[isosceles triangle,fill=black!70,minimum height=1cm,minimum width=2cm, shape border rotate=0, isosceles triangle stretches,scale=0.105] 
        at ([yshift=3.5pt,xshift=-11pt]interior.south east) {};      
      \node[isosceles triangle,fill=black!70,minimum height=1cm,minimum width=2cm, shape border rotate=90, isosceles triangle stretches,scale=0.105] 
        at ([xshift=-3.5pt,yshift=-5pt]interior.north east) {};      
      \node[isosceles triangle,fill=black!70,minimum height=1cm,minimum width=2cm, shape border rotate=-90, isosceles triangle stretches,scale=0.105] 
        at ([xshift=-3.5pt,yshift=12pt]interior.south east) {};      
      \draw[fill=gray2,draw=gray!30] 
        ([yshift=7pt,xshift=-7pt]interior.south east) rectangle 
        ([xshift=-\pgflinewidth,yshift=+\pgflinewidth]interior.south east);
      \fill[gray1!94!black]
        ([yshift=\pgflinewidth,xshift=-\pgflinewidth]interior.south east) --   
        ([yshift=7pt,xshift=-\pgflinewidth]interior.south east) --   
        ([xshift=-7pt,yshift=\pgflinewidth]interior.south east) -- cycle;  
      },
      title={\hspace*{11pt}#1-Notepad},
      height=8cm,
      left=1.5pt,
      right=8.5pt,
    }

% Miscellaneous
    % reading font
    % \usepackage{ebgaramond}
    % 首字下沉
    % \usepackage{lettrine}
    % enumerate and itemize environment
    \usepackage{enumitem}
    % Chinese support only
    \usepackage[scheme=plain]{ctex}
    % lipsum
    \usepackage{lipsum}
    % \verbatim in \macro arguments
    \usepackage{cprotect}
    % references style
    \usepackage{gbt7714}
    % show label keys
    % \usepackage{showkeys}
    % chessboard
    \usepackage{chessboard}
        \setchessboard{smallboard, showmover=false}
    % change notes
    \let\comment\relax
    \usepackage[commentmarkup=todo,
        todonotes={textsize=tiny}]{changes}
        \definechangesauthor[color=orange!30]{Iydon}
        \newcommand{\增加}[2]{\added[id=Iydon, comment={#1}]{#2}}
        \newcommand{\删除}[2]{\deleted[id=Iydon, comment={#1}]{#2}}
        \newcommand{\替换}[3]{\replaced[id=Iydon, comment={#1}]{#2}{#3}}
        \newcommand{\高亮}[2]{\highlight[id=Iydon, comment={#1}]{#2}}
        \newcommand{\注释}[1]{\comment[id=Iydon]{#1}}
