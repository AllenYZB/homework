% !Mode:: "TeX:UTF-8"
% !TEX program  = xelatex
\title[The Geometric Progression]{How to Calculate the Total Amount of Your Salary?}
\author[MA320 Group]{
    \begin{tabular}{ll}
        1. & Jin Cai \\
        2. & Zeyu Dong \\
        3. & Yi'an Yu \\
        4. & Daoyuan Lai \\
        5. & Guoyang Qin \\
        6. & Iydon Liang
    \end{tabular}
}
\date{\today}
\institute[SUSTech]{
    Department of Mathematics \\
    Southern University of Science and Technology
}

\begin{frame}
    \maketitle
\end{frame}

% \begin{frame}{Outline}
%     \tableofcontents
% \end{frame}


\section{Introduction}
\begin{frame}[<+->]{Introduction}
    \begin{exmp}
        \begin{itemize}
            \item Many jobs offer an annual cost-of-living increase to keep salaries consistent with inflation.
            \item A recent college graduate finds a position as a sales manager earning an annual salary of \$ 26,000.
            \item He is promised a 2\% cost of living increase each year.
        \end{itemize}
    \end{exmp}
    \begin{itemize}
        \item So how could we determine how much this college graduate will get after $n$, which is a natural integer, years' hard work?
        \item This question arises naturally as payment is one of the most significant factors people will consider when finding a job.
    \end{itemize}
\end{frame}



\section{The Geometric Progression}
\begin{frame}[<+->]{The Geometric Progression}
    \begin{itemize}
        \item Let's review the definition of \textbf{arithmetic progression} (Doe(2019)) first.
    \end{itemize}
    \begin{defn}
        In an \textbf{arithmetic progression}, the difference between $n^\text{th}$ term and $(n-1)^\text{th}$ term will be a constant which is known as the common difference of the arithmetic progression.
    \end{defn}
    \begin{itemize}
        \item What if the ratio of $n^\text{th}$ term to $(n-1)^\text{th}$ term in a sequence is a constant?
    \end{itemize}
\end{frame}

\begin{frame}[<+->]{The Geometric Progression}
    \begin{defn}
        A \textbf{geometric progression} is a sequence in which each term is derived by multiplying or dividing the preceding term by a fixed number called the \textbf{common ratio}.
    \end{defn}
    \begin{defn}
        If we define the initial term of a sequence $\{a_n\}$ as $a_1$, then $\{a_n\}$ is a \textbf{geometric progression} ($a_n\neq 0$ for any $n$) if and only if for any positive integer $n$,
        \[
            \frac{a_{n+1}}{a_n} = r,
        \]
        where $r$ is a constant.
    \end{defn}
\end{frame}



\section{Summation of Geometric Progression (Finite Case)}
\begin{frame}[<+->]{Summation of Geometric Progression}
    \begin{defn}
        The sum of the terms in a geometric sequence is called a \textbf{geometric series}.
    \end{defn}
    \begin{thm}
        The formula for $S_n$ is
        \[
            S_n = \frac{a_1(1-r^n)}{1-r},\quad r\neq 1,
        \]
        where $S_n$ is the geometric series.
    \end{thm}
    \begin{itemize}[<+->]
        \item How to derive this theorem?
    \end{itemize}
\end{frame}

\begin{frame}[<+->]{Summation of Geometric Progression}
    \begin{proof}
        \begin{itemize}
            \item Recall that a geometric progression is a sequence in which the ratio of any two consecutive terms is the common ratio, $r$. We can write the sum of the first $n$ terms of a geometric series as
                \begin{equation}\label{E:proof-gp-1}
                    S_n = a_1 + ra_1 + r^2a_1 + \cdots + r^{n-1}a_1.
                \end{equation}
            \item We will begin by multiplying both sides of the equation by $r$,
                \begin{equation}\label{E:proof-gp-2}
                    rS_n = ra_1 + r^2a_1 + r^3a_1 + \cdots + r^na_1.
                \end{equation}
            \item Next, if we subtract \eqref{E:proof-gp-2} from \eqref{E:proof-gp-1},
                \[
                    S_n-rS_n = (a_1 + ra_1 + r^2a_1 + \cdots + r^{n-1}a_1) - (ra_1 + r^2a_1 + r^3a_1 + \cdots + r^na_1).
                \]
        \end{itemize}
        \vspace{-.7 cm}
    \end{proof}
\end{frame}

\begin{frame}[<+->]{Summation of Geometric Progression}
    \begin{proof}
        \begin{itemize}
            \item Notice that when we subtract, all but the first term of the top equation and the last term of the bottom equation are canceled.
            \item And it is trivial to obtain the formula for $S_n$,
                \begin{equation}\label{E:gp-formula}
                    S_n = \frac{a_1(1-r^n)}{1-r},\quad r\neq 1.
                \end{equation}
            \item Notice that we have assumed that the common ratio $r$ does not equal to 1.
        \end{itemize}
    \end{proof}
\end{frame}


\section{Summation of Geometric Progression (Infinite Case)}
\begin{frame}[<+->]{Summation of Geometric Progression}
    \begin{itemize}
        \item The number of terms in infinite geometric progression will approach infinity ($n=\infty$).
        \item The formal definition of infinity has been discussed in Zhang(2018).
        \item Sum of infinity geometric progression can only be defined at the range of $-1<r<1 (r\neq 0)$ exclusive.
    \end{itemize}
    \begin{thm}
        The sum of an infinite geometric progression is
        \[
            S = \frac{a_1}{1-r}.
        \]
    \end{thm}
\end{frame}



\section{Application to the Salary Example}
\begin{frame}[<+->]{Application to the Salary Example}
    \begin{exmp}[Salary Problem]
        \begin{itemize}
            \item There will be a 102\% annual increment. So the common ratio $r$ is 1.02.
            \item If we consider the first 5 years' total amount of the salary, we could calculate $S_5$ by directly substituting $n=5$, $r=1.02$, $a_1=26,000$ into Equation~\eqref{E:gp-formula}.
            \item We will get $S_5=137,907.06$. I.e.This graduate will get \$137,907.06 after 5 years' working.
        \end{itemize}
    \end{exmp}
\end{frame}



\section{Conclusion}
\begin{frame}{Conclusion}
    \begin{itemize}[<+->]
        \item We derived the formula of the geometric series.
        \item Furthermore, with a proper choice of the common ratio, we showed that this formula can be generalized to infinite case.
        \item A real problem is analyzed to illustrate the proposed formula and answered the question in the title: by using the formula proposed in this paper, we could calculate the total amount of the salary if it has a constant multiple of the common ratio each payment period.
    \end{itemize}
\end{frame}



\section{Acknowledgments}
\begin{frame}{Acknowledgments}
    \begin{itemize}[<+->]
        \item The author would like to thank the Associate Editor and one referee for their valuable comments which have greatly improved the paper.
        \item The author's research is fully supported by the National Baby Natural Science Foundation of China(No. 1000001).
    \end{itemize}
\end{frame}

\begin{frame}
    \centering\Huge Thank you!
\end{frame}



\section{References}
\begin{frame}[c, allowframebreaks]{Reference}
    \begin{thebibliography}{9}\large
        \bibitem{Doe2019} Doe,J.(2019). The Arithmetic Progression. \emph{Mathematics for Beginners} 33:1465–1480.
        \bibitem{Smith2019} Smith,J., Smith A.J., Smith B.J.(2019). Summation of Arithmetic Progression. \emph{Journal of the Royal Baby Statistical Society, Series B} 63:111–126. 33:1465–1480.
        \bibitem{Zhang2018} Zhang,J.(2018). The Formal Discussion of Infinity. \emph{Annals of Baby Mathematics} 9:705–724.
    \end{thebibliography}
\end{frame}
