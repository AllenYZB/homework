% !Mode:: "TeX:UTF-8"
% !TEX program  = xelatex
\title{Summary For Sections 2.1 to 2.4}
\author{}
\date{}
\newgeometry{margin=2.1in, marginpar=1.9in}

\maketitle

\增加{Not from the original}{Since I'm using Linux OS, there are some compatibility problems with my PDF viewer, which require a lot of time to solve, I change the method and re-type the article by \LaTeX{}, therefore I can use ``changes'' package to achieve the same function, please forgive me.}

\注释{There is a problem with paragraph division, which means that the content in the paragraph is partly inconsistent.}

The first four sections of Chapter 2 mainly tells us the words and syntax in math writing. The author tries to emphasize \替换{Syntax error.}{how to choose words in an article}{the words to choose in an article} by quoting Halmos. Three specific suggestions are listed, which tells us to avoid technical terms, think hard about new words you must create and use old ones correctly and consistently\注释{Using list format may be more intuitive.}. Another part in this section advises that it is better to use active rather than passive \替换{Use word inaccurately.}{statement}{voice} with a comparison of two examples.

In section 2.1, the author listed the primary words for math writing categorized by different fields. After that, in section 2.2, there are \增加{Add a quantifier.}{some} lists of symbols and their explanations. Common phrases are listed in section 2.3 with both Chinese and English expressions.

In section 2.4, what we may learn is the templates for sentence \替换{Plural.}{structures}{structure}, adjectives, verbs for \替换{Use word inaccurately.}{mathematics operations}{calculating}, some other common verbs and transitional sentences. There is a standard style in a sense in math writing\注释{Unclear.}. The content, as told in the book, will be highly helpful for writing if often \替换{Double the final consonant ``r''.}{referred}{refered} to.
