% !Mode:: "TeX:UTF-8"
% !TEX program  = xelatex
\section{Summary}
If English is not your first language then writing a mathematical paper is doubly difficult, since you cannot guarantee the proper use of words even though with a large scientific and technical vocabulary. Therefore, Halmos comes up with three specific advice on how to use scientific and technical terms carefully,
\begin{enumerate}[label={\Alph*.}]
	\item Avoid using technical terms as much as possible, especially creating new ones;
	\item Be careful when thinking about the new terms you have to create by looking up the dictionary to make it as appropriate as possible;
	\item Correctly and consistently use the old terms without awkwardness.
\end{enumerate}

In addition to the terminology, you should also pay attention to the language itself. Science and technology English is generally not emotional, and its expression is relatively straightforward, which helps to make the readers easy to understand without creating too much imagination irrelevant. That is, the vocabulary using in science and technology English requires specific and sTable. Therefore, you need to accumulate basic mathematical vocabulary, mathematical symbols with their pronunciation, common phrases and language structures to better write rational mathematics papers.


\subsection{Basic Vocabulary}
Basic vocabulary is an important prerequisite because you have to grapple with a large vocabulary as you try to express your thoughts. There is a large mathematical vocabulary, but the spelling is relatively simple compared to other disciplines, and as long as you master a certain amount of mathematical vocabulary, writing an uncomplicated mathematics paper will not be too difficult.

Please see Table~2.1 to Table~2.13 in \emph{Mathematical Writing in English}\cite{2013数学之英文写作} about the basic vocabulary in different fields of mathematics.


\subsection{Symbol Overview}
Mathematical language is the universal language, therefore, when using English to write technical papers or using spoken English to express certain symbols in scientific communication, you should understand the English expression of the common mathematical symbols. In this way, you can do more with less.

Please see Table~2.14 to Table~2.17 in \emph{Mathematical Writing in English}\cite{2013数学之英文写作} about the common symbols in different fields of mathematics.


\subsection{Common Phrases}
The common mathematical phrases found in professional English textbooks or theoretically papers are very practical and require proper collection and organization because proficiently using these phrases can be beneficial to your writing.

Please see Table~2.18 to Table~2.20 in \emph{Mathematical Writing in English}\cite{2013数学之英文写作} about the common phrases.


\subsection{Language Structures}
Since mathematics writing has certain rules and formats in language structures and styles, there are many fixed modes in mathematics writing, which include some common sentence patterns, modified vocabulary, verbs in mathematical operations and transition statements.
\begin{itemize}[label={\dag}]
	\item Please see Table~2.21 in \emph{Mathematical Writing in English}\cite{2013数学之英文写作} about the common sentence patterns;
	\item Please see Table~2.22 in \emph{Mathematical Writing in English}\cite{2013数学之英文写作} about the modified vocabulary;
	\item Please see Table~2.23 in \emph{Mathematical Writing in English}\cite{2013数学之英文写作} about the verbs used in mathematical operations;
	\item Please see Table~2.24 in \emph{Mathematical Writing in English}\cite{2013数学之英文写作} about the common verbs used in mathematics writing;
	\item Please see Section~2.4.5 in \emph{Mathematical Writing in English}\cite{2013数学之英文写作} about the transition statements.
\end{itemize}



% \nocite{2013数学之英文写作,Nicholas1998Handbook}
% \bibliography{ref}
