% !Mode:: "TeX:UTF-8"
% !TEX program  = xelatex
\section{Summary}
\subsection{Proof Expression}
Mathematical proof is relatively easy in mathematics writing, but it requires clear and concise expression as well. Therefore, you can express mathematical proof skillfully after the amount of training.

Some examples are often used in proof\cite{Nicholas1998Handbook},
\begin{itemize}
	\item At the beginning of the proof, ``you should strive to keep the reader informed of where you are in the proof and what remains to be done.'' Useful phrases include
		\begin{itemize}
			\item First, we establish that \ldots
			\item Our task is now to \ldots
			\item Our problem reduces to \ldots
			\item It remains to show that \ldots
			\item We are almost ready to invoke \ldots
		\end{itemize}
	\item At the end of the proof, it is often marked by the halmos symbol □. Sometimes the abbreviation QED is used instead.
	\item If you omit part of a proof, ``it is best to indicate the nature and length of the omission'', via phrases such as the following.
		\begin{itemize}
			\item It is easy/simple/straightforward to show that \ldots
			\item Some tedious manipulation yields \ldots
			\item An easy/obvious induction gives \ldots
			\item After two applications of \ldots\ we find \ldots
			\item An argument similar to the ones used in \ldots\ shows that \ldots
		\end{itemize}
\end{itemize}


\subsection{Important Conjunction}
``Join clauses good, like a conjunction should.'' This sentence points out the importance of conjunctions in English writing, thus, we should be cautious when using conjunctions. This subsection explains the different functions of conjunctions through a series of examples, including the following functions,
\begin{itemize}
	\item Combinations:
		\begin{itemize}
			\item and, both, also, as well as, not only \ldots\ but also, apart from, in addition to, moreover, furthermore;
		\end{itemize}
	\item Implications or explanations;
		\begin{itemize}
			\item as, because, since, due to, in view of, owing to, on account of, given, it follows that, consequently, therefore, thus;
		\end{itemize}
	\item Modifications and restrictions;
		\begin{itemize}
			\item alternatively, although, though, but, whereas, by contrast, except, however, on the other hand, nevertheless, despite, in spite of, instead of, rather that.
		\end{itemize}
\end{itemize}


\subsection{Comparison}
\subsubsection{Symbols In the Sentences}
Halmos once said, ``the best notation is no notation''. That is, symbols should be used as little as possible when writing sentences that include mathematical symbols unless you have to.

Please see Section~2.7.1 in \emph{Mathematical Writing in English}\cite{2013数学之英文写作} for more examples.

\subsubsection{Definite Article and Indefinite Article}
It is difficult for speakers, who speak the languages that do not have articles, to use articles correctly in English. Therefore, there are two important rules\cite{Nicholas1998Handbook},
\begin{itemize}
	\item Do not use \emph{the} (with plural or uncountable nouns) to talk about things in general;
	\item Do not use singular countable nouns without articles.
\end{itemize}

Please see Section~2.7.2 in \emph{Mathematical Writing in English}\cite{2013数学之英文写作} for more examples.

\subsubsection{Common Misspellings or Confusions}
We often find that many words are spelling similar when learning English words, but we are not sure whether the words are used properly during the writing process. Thus we need software and accurate memory to reduce misspellings and confusion.

Please see Table~2.25 to Table~2.26 in \emph{Mathematical Writing in English}\cite{2013数学之英文写作}




\nocite{2013数学之英文写作,Nicholas1998Handbook}
\bibliography{ref}
