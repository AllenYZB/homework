% !Mode:: "TeX:UTF-8"
% !TEX program  = xelatex
In this section, we will give a proper and accurate definition of geometric progression, from which we will derive the summation formula for the geometric progression in both finite and infinite cases.

\subsection{Definition}
\begin{defnbox}{Geometric Progression\cite{Weisstein2019gp}}{gp}
    A geometric sequence is a sequence $\{a_k\}, k=0, 1, \ldots$, such that each term is given by a multiple $r$ of the previous one. Another equivalent definition is that a sequence is geometric iff\footnote{if and only if} it has a zero series bias. If the multiplier is $r$, then the $k$th term is given by
    \[
        a_k = ra_{k-1} = r^2a_{k-2} = \cdots = a_0r^k.
    \]

    Taking $a_0=1$ gives the simple special case
    \[
        a_k = r^k.
    \]
\end{defnbox}

Accordingly, we can define the arithmetic progression in advance to facilitate future introductions.
\begin{defnbox}{Arithmetic Progression\cite{Weisstein2019ap}}{ap}
    An arithmetic progression, also known as an arithmetic sequence, is a sequence of $n$ numbers $\{a_0+kd\}^{n-1}_{k=0}$ such that the differences between successive terms is a constant $d$. That is,
    \[
        a_k = a_{k-1}+d = a_{k-2}+d = \cdots = a_0+kd.
    \]

    Taking $a_0=0$ gives the simple special case
    \[
        a_k = kd.
    \]
\end{defnbox}


\subsection{Derivation}
\subsubsection{Summation Formula For the Geometric Progression}
According to the definition of the geometric progression, we have,
\begin{equation}\label{E:gp-1}
    S_n = \sum_{k=1}^{n} a_k = \sum_{k=1}^{n} a_1 r^{k-1}.
\end{equation}

Multiply both sides of the equation by $r$, then we have,
\begin{equation}\label{E:gp-2}
    rS_n = \sum_{k=1}^{n} a_1 r^{k} = \sum_{k=1}^{n} a_{k+1} = \sum_{k=2}^{n+1} a_k.
\end{equation}

Subtract equation~\eqref{E:gp-2} from equation~\eqref{E:gp-1}, we have,
\begin{equation}\label{E:gp-3}
    (1-r) S_n = a_1 - a_{n+1} = a_1(1-r^n).
\end{equation}

Therefore, we need a classification discussion, if the $r$ equals to 1, both sides of the equation cannot divide $1-r$ simultaneously, but we can derivate the formula from equation~\eqref{E:gp-1}, $S_n=na_1$.

Above all, we derivate the summation formula for the geometric progression,
\begin{equation}\label{E:gp-4}
    S_n = \begin{cases}
              na_1 & \text{if } r=1, \\
              \frac{a_1(1-r^n)}{1-r} & \text{if } r\neq 1.
          \end{cases}
\end{equation}

Moreover, if we use the form of the limit, we can simplify the formula,
\begin{equation}\label{E:gp-5}
    S_n = \lim_{q\to r} \frac{a_1(1-q^n)}{1-q}.
\end{equation}

Next we consider what happens when $n\to\infty$. $1-q$ is finite, the convergence of $S_n$ is equivalent to convergence of $\lim_{q\to r}1-q^n$. Therefore,  $S$ convergence if and only if $|r|<1$.
\begin{equation}\label{E:gp-6}
    S = \lim_{n\to\infty} S_n = \lim_{\substack{n\to\infty \\ q\to r}} \frac{a_1(1-q^n)}{1-q} = \lim_{n\to\infty}\frac{a_1(1-r^n)}{1-r} = \frac{a_1}{1-r}.
\end{equation}

In summary, we get the summation formula for the geometric progression,
\[
    S_n = \begin{cases}
              na_1 & \text{if } r=1, \\
              \frac{a_1(1-r^n)}{1-r} & \text{if } r\neq 1.
          \end{cases}
\]
and $S$ convergence if and only if $|r|<1$,
\[
    S = \lim_{n\to\infty} = \frac{a_1}{1-r}.
\]


\subsubsection{Summation Formula For the Arithmetic progression}
The summation formula for the arithmetic progression is slightly different from the geometric progression, and it relates back to a famous mathematician, Gauss. The main idea of the formula comes from
\[
    1+2+\cdots+n = \frac{n(n+1)}{2}.
\]

The proof part is left to readers as a practice question, we just provide clues below,
\begin{equation}\label{E:ap-1}
    \begin{aligned}
        S_{n+1} &= \sum_{k=0}^n a_k \\
            &= \sum_{k=0}^n (a_0+kd) \\
            &= \sum_{k=0}^n a_0 + d\sum_{k=0}^n k.
    \end{aligned}
\end{equation}
