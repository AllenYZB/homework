% !Mode:: "TeX:UTF-8"
% !TEX program  = xelatex
\newgeometry{margin=2.1in, marginpar=1.9in}



\增加{Not from the original}{Since I'm using Linux OS, there are some compatibility problems with my LibreOffice viewer, which require a lot of time to solve, I change the method and re-type the article by \LaTeX{}, therefore I can use ``changes'' package to achieve the same function. By the way, I check this article through the \href{https://docs.qq.com}{QQ Docs}.}

\注释{Some words of the section title needs to be capitalized, say ``Introduction''.}
\注释{Normally the first line of a paragraph is indented.}
\注释{Please pay attention to the articles and the verb forms.}

\section{What is \增加{An article is missing.}{a} mathematical article?}
Mathematical article including\删除{The colon is unnecessary.}{:} 1. research paper, 2. survey article or review paper, 3. referee report, 4. research proposal, 5. book review, 6. paper review, 7. reading notes and so on. Normally, a mathematical article consists of \textbf{title, abstract, introduction, main body of the article, conclusions, acknowledgments, references and appendix}. The following things we need to know, before writing an article:
\begin{itemize}
	\item \textbf{First of all}, we should \textbf{think \替换{}{about who}{who} will read} the article to make sure our readers understand the article.
	\item \textbf{Secondly}, we ought to \textbf{\替换{}{search for}{search} writing materials about what we are going to write, which may help us understand} academic terminology.
	\item \textbf{Thirdly}, we need to write \增加{}{an} article \textbf{briefly, precisely and logically}.
	\item \textbf{Fourthly}, we should \textbf{revise} our article repeatedly.
	\item \textbf{Fifthly}, we need to \textbf{take \替换{}{advice}{advise}} from people who have experience in article writing.
\end{itemize}



\section{Title}
A good title is not only \textbf{brief and precise} to tell the \textbf{main contributions or innovations} of the article but also \textbf{interesting} enough to attract readers who would read the article.
\begin{itemize}
	\item Firstly, avoid meaningless words or \增加{}{avoid} to talk in general.
	\item Secondly, be specific to reflect works in the article.
	\item Thirdly, using \增加{An article is missing.}{an} interesting title if we writing a review or expository article.
\end{itemize}



\section{Abstract}
An abstract is the heart of the article, it not only is written in \textbf{brief and precise} words\删除{The comma appears to be unnecessary.}{,} but also tell readers the \textbf{purpose, method and result of the article and make sure most of \增加{}{the} readers understand the abstract}. We should write the abstract after finishing the first edition of the article, avoid using sentences from \增加{}{the} introduction and conclusions. Normally, an abstract only contains 100 to 300 words.



\section{introduction}
An introduction should give \替换{}{an}{the} accurate background of the study problem objectively, including \textbf{history of the study, motivation and purpose} of the article and the \textbf{results} of the article.

\textbf{In the beginning}, we should mention the problem we meet in the study.

\textbf{In the middle}, we should define the problem and explain what attempts to do, not only summarize the results, progress achieved and problem left, but also mention our plan for the unsolved problem.

\textbf{In the end}, we give \替换{}{an}{the} outline of the article structure.



\section{The main body of the article}
If \增加{}{the} article focuses on theoretic math, the main body consists of \textbf{preliminaries, main result, the outline of proofs and extensions}. In \增加{}{the} preliminaries part, we talk about notation and lemma. In main results part, conclusion is mentioned and discussed.

If \增加{}{the} article focuses on computational math or applied math, the main body consists of \textbf{the problem or governing equations, the numerical method or experimental method, theoretical analysis, numerical results and conclusions}.

When we doing numerical \替换{}{experiments}{experiment}, we should \textbf{explain details} such as parameters and \替换{}{methods}{method} we use firstly. Next, using \textbf{moderate figures and tables} to contract may explain the results better. Thirdly, using \textbf{good examples}, which is widely used, may be persuasive. At last, \textbf{explanations about innovations} of the numerical results is necessary.



\section{Conclusions}
In the conclusions part, the following information we should concern.
\begin{enumerate}
	\item Provide a \textbf{summary of the main contribution} briefly, emphasize the \替换{}{importance}{important} of the results and try not to use repeated words.
	\item Explain the questions mentioned in introduction.
	\item Discuss the \textbf{general problems} which not mentioned in the article.
	\item Explain the \textbf{implications} of our research.
	\item Identify the next step or look to \增加{}{the} future.
\end{enumerate}



\section{Acknowledgments}
Acknowledgments \替换{}{are}{is} the part where we thank \删除{}{to} \textbf{people and institutions} who helped us to finish the article. We also need to mention \textbf{reviewers} if they give us significant advice which \替换{}{helps}{help} us revise the article.



\section{Reference}
Reference is the part where we put the \textbf{papers or books} which are used in the article. The paper in reference contains the following information: authors, article title, journal title, volume number, issue number, publication year and page numbers. The book in reference contains the following information: author, book title, publisher, edition information, publication city. Conference paper in reference also \替换{}{needs}{need} to mention editors. Dissertation in reference \替换{}{has}{have} to show the institution information.



\section{Appendix}
Appendix is the part where we put \textbf{long and complex mathematical formulas of proof} and some \textbf{relevant lemmas} but not concerned in the paper.



\section{Others}
\begin{enumerate}
	\item Name: take the responsibility of the article
	\item Date: get the data when we revise our article
	\item Keywords and subject classification: make your article easy to get found.
	\item Chapter title: interested readers and make the article logical.
\end{enumerate}



\section{Conclusion}
We not only acquire the basic knowledge of writing but also need to \textbf{read western mathematician papers} to enlarge our English mathematical vocabulary.
