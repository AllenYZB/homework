% !Mode:: "TeX:UTF-8"
% !TEX program  = xelatex

% Hyperref
    % hyperref
    \usepackage{hyperref}
        \hypersetup{
            pdfauthor={Iydon},
            pdfcreator={Microsoft® Word 2017},
            pdfproducer={Microsoft® Word 2017},
        }

% Math
    % symbols and environment
    \usepackage{amsmath,amssymb}
        \newcommand{\dd}{\mathrm{d}}
        \newcommand{\diff}[2]{\frac{\dd #1}{\dd #2}}
        \newcommand{\tdiff}[2]{\tfrac{\dd #1}{\dd #2}}
        \newcommand{\RR}{\mathbb{R}}
        \newcommand{\EE}{\mathbb{E}}
        \newcommand{\vecw}{\symbf{w}}
        \newcommand{\vecx}{\symbf{X}}
        \newcommand{\vecy}{\symbf{Y}}
        \newcommand{\logit}{\mathrm{logit}}
        \newcommand{\var}{\mathrm{Var}}
        \newcommand{\小括号}[1]{\left(#1\right)}
        \newcommand{\中括号}[1]{\left[#1\right]}
        \newcommand{\大括号}[1]{\left\{#1\right\}}
    % font
    \usepackage{unicode-math}
        \unimathsetup{math-style=ISO, bold-style=ISO, mathrm=sym}
        \setsansfont{FiraGO}[BoldFont=* SemiBold, Numbers=Monospaced]
        \setmathfont{Fira Math Regular}

% Header and Footer
    % geometry
    \usepackage{geometry}
        % \geometry{margin=1in}
    % fancy header
    \usepackage{tikz}
        \usetikzlibrary{ducks,external}
    \usepackage{fancyhdr}
        % \fancyhead[L]{}
        % \fancyhead[C]{Iydon Liang, 11711217}
        % \fancyhead[R]{}
        % \fancyfoot[C]{%
        %     \tikzset{external/export=false}%
        %     \shuffleducks
        %     \begin{tikzpicture}[scale=0.5]
        %         \duck[signpost=\scalebox{0.6}{\thepage},\randomhead]
        %     \end{tikzpicture}
        % }
        \renewcommand{\headrulewidth}{0.4pt}
        \pagestyle{fancy}

% Listings
    % Python
    \usepackage{tcolorbox}
        \tcbuselibrary{listings,breakable}
        \newtcblisting{pylist}[1]{
            listing options={language=Python,numbers=left,
                numberstyle=\tiny\color{red!75!black},breaklines,
                basicstyle=\small,keywordstyle=\color{blue!85!black},
                commentstyle=\color{white!75!black}\textit},
            title=\texttt{#1},listing only,breakable,
            left=6mm,right=6mm,top=2mm,bottom=2mm,
            colback=violet!5!white,colframe=violet!75!black}
        \newtcbinputlisting{\Python}[2]{%
            listing options={language=Python,numbers=left,
                numberstyle=\tiny\color{red!75!black},
                basicstyle=\small,keywordstyle=\color{blue!85!black},
                commentstyle=\color{white!75!black}\textit,breaklines},
            listing only,breakable,
            listing file={#2},title=\texttt{#1},
            left=6mm,right=6mm,top=2mm,bottom=2mm,
            colback=violet!5!white,colframe=violet!75!black}
        \newtcbinputlisting{\MATLAB}[2]{%
            listing options={language=MATLAB,numbers=left,
                numberstyle=\tiny\color{red!75!black},
                basicstyle=\small,keywordstyle=\color{blue!85!black},
                commentstyle=\color{white!75!black}\textit,breaklines},
            listing only,breakable,
            listing file={#2},title=\texttt{#1},
            left=6mm,right=6mm,top=2mm,bottom=2mm,
            colback=violet!5!white,colframe=violet!75!black}
        \newtcbinputlisting{\Cpp}[2]{%
            listing options={language=C++,numbers=left,
                numberstyle=\tiny\color{red!75!black},
                basicstyle=\small,keywordstyle=\color{blue!85!black},
                commentstyle=\color{white!75!black}\textit,breaklines},
            listing only,breakable,
            listing file={#2},title=\texttt{#1},
            left=6mm,right=6mm,top=2mm,bottom=2mm,
            colback=violet!5!white,colframe=violet!75!black}
        \tcbuselibrary{xparse}
            \DeclareTotalTCBox{\verbbox}{ O{orange} v !O{} }
                {fontupper=\ttfamily,nobeforeafter,tcbox raise base,
                 arc=0pt,outer arc=0pt,top=0pt,bottom=0pt,left=0mm,
                 right=0mm,leftrule=0pt,rightrule=0pt,toprule=0.3mm,
                 bottomrule=0.3mm,boxsep=0.5mm,bottomrule=0.3mm,boxsep=0.5mm,
                 colback=#1!10!white,colframe=#1!50!black,#3}{#2}
        \DeclareTotalTCBox{\commandbox}{ s v }
                {verbatim,colupper=white,colback=black!75!white,colframe=black}
                {\IfBooleanTF{#1}{\textcolor{red}{\ttfamily\bfseries >> }}{}
                    \lstinline[language=sh,morekeywords={python,python2,python3},keywordstyle=\color{blue!35!white}\bfseries]^#2^}
        \tcbuselibrary{theorems}
        \newtcbtheorem{theobox}{Theorem}%
            {colback=white,colframe=orange!90!black,fonttitle=\bfseries,%
             arc=2mm,separator sign={\ $\blacktriangleright$}}{theo}%
        \newtcbtheorem{defnbox}{Definition}%
            {colback=white,colframe=blue!90!black,fonttitle=\bfseries,%
             arc=2mm,separator sign={\ $\blacktriangleright$}}{defn}%
        \newtcbtheorem{statebox}{Statement}%
            {colback=white,colframe=orange!90!black,fonttitle=\bfseries,%
             arc=2mm,separator sign={\ $\blacktriangleright$}}{state}%

% Miscellaneous
    % graphic
    \usepackage{graphicx,float}
    % reading font
    \usepackage{ebgaramond}
    % Chinese support only
    \usepackage[scheme=plain]{ctex}
    % \usepackage[scheme=plain]{ctex}
    % lipsum
    \usepackage{lipsum}
    % \verbatim in \macro arguments
    \usepackage{cprotect}
    % references style
    \usepackage{gbt7714}
    % show label keys
    % \usepackage{showkeys}
    % change notes
    \let\comment\relax
    \usepackage[commentmarkup=todo,
        todonotes={textsize=tiny}]{changes}
        \definechangesauthor[color=violet!30]{Iydon}
        \newcommand{\增加}[2]{\added[id=Iydon, comment={#1}]{#2}}
        \newcommand{\删除}[2]{\deleted[id=Iydon, comment={#1}]{#2}}
        \newcommand{\替换}[3]{\replaced[id=Iydon, comment={#1}]{#2}{#3}}
        \newcommand{\高亮}[2]{\highlight[id=Iydon, comment={#1}]{#2}}
        \newcommand{\注释}[1]{\comment[id=Iydon]{#1}}
