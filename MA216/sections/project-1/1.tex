% !Mode:: "TeX:UTF-8"
% !TEX program  = xelatex
\section{Introduction}
LPPL Model is used to describe a super-exponential (power law) accelerating behavior of asset price when just before the bubbles burst and crushes or rebounds appearing. The characteristic of abnormal increasing bubble price is a strong upward curvature with seriously unstable oscillation, which implies the price goes upward by growth rates at any time, instead of a constant growth rate which is the simple exponential increasing process\cite{R:1}. This model is closer to the real market because the asset price in the real market will never grow with an ideal constant rate and this model accounts for the imitation and herding mechanisms and positive feedback among investors, so it can be a more accurate prediction indicator\cite{R:2}.
\subsection{Definition of Bubbles}
The asset price will be over-estimated and become irrational high after a series of price increases because more and more investors follow the trend to buy this popular asset, as a result, the market price of the asset will beyond its real price. The more the asset's owner expects to earn from holding or selling that asset, the higher its price. When the price reaches a relatively highest position, there is no follower who is willing to hold the asset with a higher price, then the holders start to lost confidence and leave the asset gradually, as the result, the difference between the bid and call becomes too large to remain balance, finally the bubble burst. If this unbalance between selling and holding is broken, the bubble will disrupt and a crash follows which means the price will suddenly decay after reaching the highest price. Many factors may contribute to economic bubbles, such as speculation, easy credit, finance innovation.


\subsection{Positive Feedback}
In the stock market, any deviation of asset price should be ultimately traced back to the behavior of investors. it is the buying and selling decisions of investors that push prices up and down, the mechanisms of positive feedback on price is that if the recent observation of the asset price is moved up, then price will follow the tendency to keep on moving up, as a result, the asset price will be grow super-exponentially.

The positive feedback leads to speculative behaviors which means that the more and more trends followers emerge in the market along with large investments accumulated by Youssefmir (1998) \cite{R:3}. It will cause the market value larger than its real value presenting a greater-exponential upward tendency. At the same time, the system will become increasingly sensible until the crash.


\subsection{Imitation and Herding Behavior}
Herding effect theorem is the basis of positive feedback which is popular in many economics phenomena. It happens because investors are not able to attain the whole picture of a market, they can just make a buying or selling decisions by analyzing recent observations or following those expert and rational investors. The less information gained from the market, the larger the probability to follow others like a herd.

There was a mimetic contagion model of investors in the stock markets developed by Orlean (1989)\cite{R:4}. The simplest version is called the Urn model. Under stimulating this model with balls, it turns out that bubble crushes are the consequences of the imitative behavior among investors.


\subsection{Formula}
For the original formula of LPPL model by Didier Scornette,
\begin{equation}\label{F:lPPL}
\ln p(t) = A − B(t_c − t)^m + C(t_c − t)^m\cos(\omega\ln(t_c − t) + \varphi))
\end{equation}
There are seven unknown parameters in the formula above, where $p(t)$ denote the price of asset at time $t$,
\begin{itemize}
    \item $A$ is the logarithm price of greater-than-exponential bubble price when it approaches to $t_c$, $A > 0$.
    \item $B$ denotes the direction of price change is either an upward gain ($B < 0$) or a downward loss ($B > 0$).
    \item $C$ includes the degree of logarithm periodic oscillation.
    \item $m$ indicates the extent of accelerating, which depends on the proportion of rational investors and herding followers ($0 < m < 1$).
    \item $t$ is a certain time interval in the past that is any time before the bubble burst.
    \item $\omega$ is the angular frequency of oscillation during bubbles, it exists when investors show consistency on an investment strategy.
    \item $\varphi$ is the initial phase, $0 < \varphi < 2\pi$.
\end{itemize}

A more detailed explanation about seven parameters: In our following study, we suppose the price remains finite even at $t_c$ ,$B < 0$ for upward accelerating price while $B > 0$ for downward accelerating price and the accelerating extent near the critical point $m$ should be $0 < m < 1$. If $m$ is too close to 0, it implies a relatively stationary bubble while accelerating suddenly approaching the critical time, If m is too close to 1, it presents no price-increasing tendency. When it comes to $w$, by spectrum analysis on residual studied by Press, W.H., Teulolsky (1994), logogram frequency is about 1.1, then the corresponding angular frequency is about 7 \cite{R:5}. We combine these theoretical analyses and experimental results as presumptions and put into the fitting model to ensure an explainable super-exponential acceleration behavior of the bubble price.


\subsection{Application}
Many systems present similar super-exponential growth regimes, which can be described mathematically by power law growth. For example, planet formation in solar systems by runaway accretion of planetesimals, rupture and material failures, nucleation of earthquakes modeled with the slip-and-velocity, models of micro-organisms interacting through chemotaxis aggregating to form fruiting bodies, the Euler rotating disk, and so on.

LPPL has presented pretty good examples of predicting the crisis risk, such as the Oil price bubble in early July 2008 and the bubble burst on the Shanghai stock market in early August 2009\cite{R:6}. Therefore, It has great significance to forecast the critical time of bubble burst with a certain probability. And nowadays LPPL has been used commonly in the field among financial, earthquake, biochemistry.


\subsection{Conclusion}
we would like to use and revise the LPPL Model by analyzing the possible factors and mechanisms behind its high frequency oscillation and super exponential behaviors. In this sense, we might take other interactions or factors into account to revise the LPPL Model in order to provide a more accurate and precise indicator of crisis risk.
