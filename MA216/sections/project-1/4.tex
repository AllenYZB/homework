% !Mode:: "TeX:UTF-8"
% !TEX program  = xelatex
\section{View of Ted Talk of Didier Scornette}
Professor Didier Scornette has pioneered in studying in the critical time of financial crisis, they started the Financial Crisis Observatory in order to study if critical time of financial bubbles can be detected in advance and developed a theory called ``Dragon Kings'', which compared those extremely special and large events to ``dragon'' and ``kings''. In the real market, the asset prices normally move up in a natural exponential way, however, the extreme situation is inevitable and not rarely although it only possesses 1 percentage. The financial crisis like ``Great Recession'' brought incredible tragedy to the world market and risk management tools at present are not strong enough to diagnose these outliers, even will misguide the professional analysts. Thus, it is essential to find out a powerful theorem and model to make it controllable and predictable.
Fortunately, the Scornette’s group has identified that the information about the critical time of the system eventually burst is contained in the early development of the previous formal super-exponential growth. The normal prices move upward with a greater-exponential growth tendency. The next growth is pushed forward by positive feedback. The Dragon King theorem provides a possible diagnostics method of crisis for investors, to some extent, agents can prepare and take measures in advance to prevent the crisis like the Great Recession occurring again.

It can be applied to many other fields where run the similar regime that affected by positive feedback and contains the extreme consequence, such as to find the critical time interval of machine rupture occurs depending slight noise of emission, the precursor sign of baby maturation, glacier collapse, blockbusters, earthquake forecasting and so on. They have successfully predicted lots of financial bubbles in the past 20 years applying this model, taking the Chinese bubbles crushed in December 2007 while rebounded in August 2009 as examples.

Although the prediction method presented in the TED Talks seems to be an excellent way to reveal the abnormal phenomena, the explanatory power of a formula can be questioned due to seven unknown parameters within a complicated formula of LPPL. In the course of dealing with practical problems, fitting data and returning a group of parameters is not easy. It can not explain the real meanings of the value of all parameters, analysts can only refer to thousands of outputs from successful experiments. It is the secret of model fitting that matters the excellent prediction consequence.
