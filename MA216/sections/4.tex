% !Mode:: "TeX:UTF-8"
% !TEX program  = xelatex
\title{Assignment 4}


\section{Question 1}
\begin{statebox}{}{question-1}
    Why is the denominator $M-1$ instead of $M$?
    \[
    	\sigma_M^2 := \frac{1}{M-1}\sum_{i=1}^{M} (\xi_i-\mu_M)^2
    \]
\end{statebox}


简单说是因为共线性,$n$个样本的自由度为$n$,但是样本都减去样本均值后自由度就变为$n-1$,所以分母上应为$n-1$。证明思路为
\begin{align*}
	E\left[\frac{1}{n}\sum_{i=1}^{n}\left(X_i-\bar{X}\right)^2\right] &= E\left[\frac{1}{n}\sum_{i=1}^{n}\left((X_i-\mu)-(\bar{X}-\mu)\right)^2\right] \\
	&= E\left[\frac{1}{n}\sum_{i=1}^{n}\left(X_i-\mu\right)^2\right] - E\left[\frac{1}{n}\sum_{i=1}^{n}\left(\bar{X}-\mu\right)^2\right] \\
	&= \sigma^2 - \frac{1}{n}\sigma^2 = \frac{n-1}{n}\sigma^2
\end{align*}

因此我们得到满足$E[S^2]=\sigma^2$的无偏估计
\[
	S^2 = \frac{1}{n-1}\sum_{i=1}^{n}\left(X_i-\bar{X}\right)^2.
\]



\section{Question 2 and 3}
\begin{statebox}{}{question-2}
    Try both cases in computer simulations and what is your observation?
    Prove that the $M-1$ case is better in the statistical sense.
\end{statebox}

运行程序如下,其中共进行65536次实验,每次生成服从正态分布$N(0, 1)$的1024组数据进行运算。无偏估计较好的实验次数为33347,超过实验总数的一半以上。且多次运行程序结果相差不大,无一例外地无偏估计所得结果较好,所以使用$M-1$作为分母在统计学上是更好的选择。
\Cpp{Simulations}{code/4.cpp}



% \bibliography{ref}
